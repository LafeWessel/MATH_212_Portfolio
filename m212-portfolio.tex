\documentclass[11pt,titlepage]{article}		% The percent symbol in your code starts a comment.  The comment ends at the next linebreak.

\usepackage[english]{babel} 		% Packages add functionality and style conventions to your documents. Don't edit this section!
\usepackage{fullpage}				% Eliminates wasted space
\usepackage[utf8]{inputenc}			% Necessary for character encoding
\usepackage{amsmath, amssymb,amsthm}% Required math packages
\usepackage{graphicx}				% For handling graphics
\usepackage[colorinlistoftodos]{todonotes}	% For the fancy "todo" stuff
\usepackage{hyperref}				% For clickable links in the final PDF
\usepackage{enumitem}
%\usepackage{titling}				% To take less space at the top of the page with the title
%\setlength{\droptitle}{-2cm}
%\pretitle{\Large\scshape}%{\begin{flushright}\Large\scshape}
%\posttitle{\par\end{flushright}}
%\preauthor{\large\scshape}
%\postauthor{\par\end{flushright}}
%\predate{\large\scshape}
%\postdate{\par\end{flushright}}
\linespread{1.5}

\newcommand{\set}[1]{\left\{ {#1} \right\}}
\newcommand{\setof}[2]{{\left\{#1\,\colon\,#2\right\}}}

\def\rubric{\textbf{Evaluation:} \makebox[0.75in]{\hrulefill}

\vspace{.3in}

\textbf{Opening:} \makebox[0.75in]{\hrulefill}

\vspace{.3in}

\textbf{Logical Correctness:} \makebox[0.75in]{\hrulefill}

\vspace{.3in}

\textbf{Reasons:} \makebox[0.75in]{\hrulefill}

\vspace{.3in}

\textbf{Use of Notation:} \makebox[0.75in]{\hrulefill}

\vspace{.3in}

\textbf{Clarity and Writing:} \makebox[0.75in]{\hrulefill}

\vspace{.3in}

\textbf{\LaTeX\ Formatting:} \makebox[0.75in]{\hrulefill}

\vspace{.3in}

\textbf{Stating the Conclusion:} \makebox[0.75in]{\hrulefill}

\vspace{.3in}

\textbf{Other Comments:}

\vspace{1in}

}

% Type `\C' for the complex numbers, `\H' for the quarternions, etc.
\def\C{{\mathbb C}}
\def\H{{\mathbb H}}
\def\Z{{\mathbb Z}}
\def\Q{{\mathbb Q}}
\def\R{{\mathbb R}}
\def\N{{\mathbb N}}


%\Alpha{homeworkresults}

\theoremstyle{theorem}
\newtheorem{theorem}{Theorem}
\renewcommand*{\thetheorem}{\Roman{theorem}}
%\setcounter{theorem}{2}
\newtheorem{lemma}[theorem]{Lemma}
\newtheorem{prop}[theorem]{Proposition}
\newtheorem{claim}[theorem]{Claim}
\newtheorem{example}[theorem]{Example}
\newtheorem{conjecture}[theorem]{Conjecture}




\title{\sc Math 212 Portfolio}

\author{Lafe Wessel}

\date{Draft date: \today}

\begin{document}

\maketitle


\noindent\textbf{Changelog:} \emph{List the changes you've made since the last draft, with special attention paid to problems that have received significant revisions since the last draft (i.e., more than fixing typos). If there is any additional information you'd like me to consider as I review this submission, please say so now.}

\begin{enumerate}
\item Reworked Conjecture V
\item Added Conjecture VI
\end{enumerate}

\noindent\textbf{Instructions:} Each of the problems below is/will be presented as a conjecture. Each conjecture asks you to prove or disprove the conjecture, possibly along with some additional directions. 

\bigskip

\begin{itemize}  
	\item If the conjecture is true, your job is to write a complete proof for the proposition. If there are multiple parts, you should consider each part in turn.
	\item If it is false, you should provide a counterexample plus make reasonable modifications to the stated conjecture so that a new proposition is true. Then, write a complete proof of your new proposition. You may want to run your new proposition by me before trying to write a proof--this is allowed and encouraged!
\end{itemize}


\noindent\textbf{Academic Honesty Policy:}
The portfolio is an independent project in which no outside resources or collaboration is allowed. You may not ask other professors or discuss the problems with anyone besides me. You should not discuss even which problem you chose. Violation of this policy is grounds for failing the course. The point is that you need to be confident and competent in writing proofs for future courses.






\clearpage

\begin{conjecture}
	Let $A$ and $B$ be subsets of some universe $\mathcal{U}$.
	Then:
	\begin{enumerate}
		\item $A\setminus (A\cap \overline{B}) = A\cap B$
		\item $\overline{(\overline{A}\cup B)} \cap A = A\setminus B$
		\item $(A\cup B)\setminus A = B\setminus A$
		\item $(A\cup B) \setminus B = A\setminus (A\cap B)$
	\end{enumerate}
\end{conjecture}

All four statements of Conjecture I will be proved to be equal through the set equivalence method: demonstrating that the left side is a subset of the right side and that the right side is a subset of the left side.

Conjecture I.1.
\begin{proof}
Let $x\in A\setminus{(A \cap\overline{B})}$. Thus, $x$ must be an element of $A$ and not an element of $A\cap\overline{B}$, that is $x\in A$ and $x\not\in A\cap\overline{B}$. If $x$ is not in $A\cap\overline{B}$, then $x$ must not be in $\overline{B}$, meaning $x\in B$. Thus, $x\in A\cap B$, and, furthermore, $A\setminus{(A\cap\overline{B})}\subseteq A\cap B$.

Let $x\in A\cap B$. For this to be true, $x$ must be in both $A$ and $B$; $x\in A$ and $x\in B$. Thus, $x$ cannot be in the complement of $B$, $\overline{B}$. Since $x\not\in\overline{B}$, $x\not\in A\cap\overline{B}$. Because $x\in A$ and $x\not\in A\cap\overline{B}$, $x \in A\setminus{(A\cap\overline{B})}$. Thus, $A\cap B \subseteq A\setminus{(A\cap\overline{B})}$.

Therefore, since $A\setminus{(A\cap\overline{B})}\subseteq A\cap B$ and $A\cap B \subseteq A\setminus{(A\cap\overline{B})}$, $A\setminus{(A\cap\overline{B})} = A\cap B$.
\end{proof}

Conjecture I.2
\begin{proof}
Let $x \in \overline{(\overline{A}\cup B)}\cap A$. Thus, $x \in A$ and $ x\in \overline{(\overline{A}\cup B)}$. This means that $x$ is not in the complement of $\overline{(\overline{A}\cup B)}$, which is $\overline{A} \cup B$. Thus, $x\not\in\overline{A}\cup B$. Consequently, $x \not\in \overline{A}$ and $x\not\in B$. If $x \not\in\overline{A}$, then $x\in A$. Since $x\in A$ and $x\not\in B$, $x\in A\setminus B$. Thus, if $x \in A\setminus B$ and $x \in \overline{(\overline{A}\cup B)}\cap A$, $\overline{(\overline{A}\cup B)}\cap A \subseteq A \setminus B$.

Let $x \in A\setminus B$. If this is true, then $x \in A$ and $x\not\in B$. It follows that $x\not\in\overline{A}$ and, consequently, that $x\not\in\overline{A}\cup B$. If $x\not\in\overline{A}\cup B$, then $x$ must be in the complement, being $\overline{(\overline{A}\cup B)}$. Since $x \in A$ and $x \in \overline{(\overline{A}\cup B)}$, then $x\in \overline{(\overline{A}\cup B)}\cap A$. Thus, $A\setminus B \subseteq \overline{(\overline{A}\cup B)}\cap A$.

Therefore, since $\overline{(\overline{A}\cup B)}\cap A \subseteq A \setminus B$ and $A\setminus B \subseteq \overline{(\overline{A}\cup B)}\cap A$, $\overline{(\overline{A}\cup B)}\cap A = A \setminus B$.
\end{proof}

Conjecture I.3
\begin{proof}
Let $x\in (A \cup B)\setminus A$. For $x$ to be an element, it must be in $A\cup B$ and not in $A$. Thus, $x\in A\cup B$ and $x\not\in A$. Since $x\not\in A$, $x$ must be a part of $B$ for $A\cup B$ to be true: $x\in B$. If $x\not\in A$ and $x\in B$, then $x\in B\setminus A$. If $x\in (A \cup B)\setminus A$ and $x\in B\setminus A$, then $(A \cup B)\setminus A \subseteq B \setminus A$.

Let $x\in B\setminus A$. Thus, $x$ must be in $B$ and not in $A$. Accordingly, $x\in B$ and $x\not\in A$. It follows that $x \in A\cup B$, and, consequently, that $x \in (A \cup B)\setminus A$ since $x \not \in A$. Thus, because $x\in B \setminus A$ and $x \in (A \cup B)\setminus A$, $B\setminus A \subseteq (A\cup B)\setminus A$.

Therefore, since $(A \cup B)\setminus A \subseteq B \setminus A$ and $B\setminus A \subseteq (A\cup B)\setminus A$, $(A \cup B)\setminus A = B \setminus A$.
\end{proof}

Conjecture I.4
\begin{proof}
Let $x \in (A \cup B) \setminus B$. It follows that $x \in A\cup B$ and $x\not\in B$. This means that $x\in A$ since $x\in A\cup B$ is true. Thus, $x\not\in A\cap B$. Because $x\in A$ and $x\not\in A \cap B$, $x\in A\setminus (A\cap B)$ is true. Thus, since $x\in (A \cup B) \setminus B$ and $x\in A\setminus (A\cap B)$, $ (A\cup B) \setminus B \subseteq A\setminus (A \cap B)$.

Let $x \in A\setminus(A\cap B)$. Thus, $x \in A$ and $x \not \in A \cap B$, meaning that $x \not\in B$. If follows that $x \in A \cup B$ and, since $x \not \in B$, $ x\in (A \cup B)\setminus B$ are both true. Thus, if $x \in A\setminus(A\cap B)$ and $ x\in (A \cup B)\setminus B$ are true, then $A\setminus(A\cap B) \subseteq (A \cup B)\setminus B$ is true.

Therefore, since $ (A\cup B) \setminus B \subseteq A\setminus (A \cap B)$ and $A\setminus(A\cap B) \subseteq (A \cup B)\setminus B$ are true, it must be true that $ (A\cup B) \setminus B = A\setminus (A \cap B)$.
\end{proof}

\clearpage
\begin{conjecture}
Define $f: \N\setminus\set{0}\to\Z$ as follows: for each $n\in \N\setminus \set{0}$,
\[
f(n) = \frac{1+(-1)^n (2n-1)}{4}.
\]
Then $f$ is a bijection.
\end{conjecture}

Conjecture II will be proved to be a bijection by first showing that it is an injection and subsequently that it is a surjection. Proving the function injective will be done by demonstrating that $f(x)=f(y) \to x=y$, and proving the function surjective will be done by demonstrating that every value of the codomain is reflected by a value of the domain.

We notice that when $n$ is even, the result is positive, and when $n$ is odd, the result is negative. This is because $(-1)^1 = -1, (-1)^3=-1,...$ and $(-1)^2=1, (-1)^4=1,..$. Thus, we must evaluate the function both when $n$ is even and when $n$ is odd.


\begin{proof}
First, we show that $f$ is an injection.

When $n$ is even, we show that $f(x)=f(y) \to x=y$, and we know that $(-1)^n = 1$.
\begin{align}
f(x)&=\frac{1+(1)(2x-1)}{4}\\
f(y)&=\frac{1+(1)(2y-1)}{4}\\
\frac{1+(1)(2x-1)}{4} &= \frac{1+(1)(2y-1)}{4}\\
1+(1)(2x-1) &= 1+(1)(2y-1)\\
(1)(2x-1) &= (1)(2y-1)\\
2x-1 &= 2y-1\\
2x&=2y\\
x&=y
\end{align}
When $n$ is odd, we show that $f(x)=f(y) \to x=y$, and we know that $(-1)^n = -1$.
\begin{align}
f(x)&=\frac{1+(-1)(2x-1)}{4}\\
f(y)&=\frac{1+(-1)(2y-1)}{4}\\
\frac{1+(-1)(2x-1)}{4} &= \frac{1+(-1)(2y-1)}{4}\\
1+(-1)(2x-1) &= 1+(-1)(2y-1)\\
(-1)(2x-1) &= (-1)(2y-1)\\
-2x+1 &= -2y+1\\
-2x&=-2y\\
x&=y
\end{align}
Therefore, since $f(x)=f(y) \to x=y$ both when $n$ is even and odd, we know that the function $f(n) = \frac{1+(-1)^n (2n-1)}{4}$ is an injection.


Next, we show that $f$ is a surjection.

We must solve for $y$ in $f(x) =y$ to demonstrate that the codomain has at least one corresponding value of the domain in order for $f(n)$ to be surjective. 

When $n$ is even, we know that $(-1)^n = 1$. Solving for $x$, we find that $x=2y$. Thus, since the function $2y=x$ is linear, having all possible values of $x$ and $y$ as its domain and codomain, $f(n)$ is surjective when $n$ is even.

When $n$ is odd, we know that $(-1)^n=-1$. Solving for $x$, we find that $x=-2y+1$. Thus, since the function $-2y+1=x$ is linear, having all possible values of $x$ and $y$ as its domain and codomain, $f(n)$ is surjective when $n$ is odd.

Since $f(n)$ is surjective both when $n$ is even and when it is odd, we know that the function is surjective.

Therefore, since $f(n)$ has been demonstrated to be both injective and surjective, the function is a bijection.
\end{proof}
\clearpage


\begin{example}
For the following example, choose two of the four problems to do. Exactly one of your choices should be a combinatorial proof.

\begin{enumerate}
    \item (Combinatorial) For $n\ge 1$,
    \[
        \sum\limits_{k=0}^n k \binom{n}{k} = n 2^{n-1}.
    \]
    
    \item (Combinatorial) For $0\le k \le n$,
    \[
        \sum\limits_{m=k}^n \binom{m}{k} = \binom{n+1}{k+1}.
    \]
    \item Consider the alphabet $\{a,b,c,d,e,f\}$ and make words without repetition of letters allowed.
        \begin{enumerate}
            \item How many six-letter words are there?
            \item How many words begin with \textit{d} or \textit{e}?
            \item How many words end in \textit{b} or \textit{a}?
            \item How many words begin with \textit{d} or \textit{e} and end in \textit{b} or \textit{a}?
            \item How many have first letter neither \textit{d} nor \textit{e} and last letter neither \textit{b} nor \textit{a}?
        \end{enumerate}
    \item We wish to improve upon the ogre's distribution of 43 cupcakes to 12 baby mice by ensuring that every baby mouse gets at least \textit{two} cupcakes. How many ways are there to accomplish this?
\end{enumerate}

\end{example}
\clearpage
\begin{proof}
Proof 1

The summation 
    \[
        \sum\limits_{k=0}^n k \binom{n}{k} = n 2^{n-1}
    \]
when $n\ge 1$, shall be demonstrated to be true as a way of counting sets. The cardinality of the power set of a set with cardinality $n$ is $2^n$. This number is the total number of ways to include or exclude any elements from the set. Thus, $2^{n-1}$ is the number of ways to choose any number of elements from a set with cardinality $n-1$. Multiplying this by $n$ is the ways to arrange that set for each element in $n$. Thus, the right-hand side of the equation can be thought of as counting the ways to choose elements from a set for each way that we can remove one element.

The sum of each row of Pascal's Triangle is equal to $2^{depth}$. Thus, 
    \[
        \sum\limits_{k=0}^n \binom{n}{k} = 2^{n},
    \]
since each element of Pascal's Triangle is equal to $\binom{n}{k}$ for a given $n$ and $k$ when $n \ge 1$ and $n \ge k \ge 0$. Multiplying each $\binom{n}{k}$ by $k$ counts the ways to choose $k$ from a set of $n$ elements for each element being chosen, $k$. (i.e. For each of $k$ sweaters, how many ways are there to choose $k$ sweaters from a wardrobe of $n$ sweaters?)  This keeps the symmetry of Pascal's Triangle and when summed across a row, will equal $n\cdot2^{n-1}$. Thus, the left-hand side of the equation can be thought of as counting the ways that we can choose $k$ elements from $n$ for each element that we choose, and summing them until we are choosing the entire set.

This proof answers the question how many ways can I choose one element, then for each of 2 options I choose 2 from the set, then for 3 options I choose 3 from the set, all the way until for each $n$ options I choose $n$ elements from the set containing $n$ elements. Or, I could count the total ways to choose from a set of $n-1$ elements for each of $n$ elements, being $n2^{n-1}$.
\end{proof}
\clearpage
\begin{proof}
Proof 3
\begin{enumerate}[label=(\alph*)]
\item There are 6 options for the first letter, 5 for the second, 4 for the third... for a total of $6! = 720$ options for 6-letter words.
\item There are 2 options for the first letter, 5 for the second, 4 for the third... for a total of $2\cdot5! = 240$ words that start with $d$ or $e$.
\item This is essentially $(b)$ but backwards. There are 2 options for the last letter, 5 options for the penultimate... for a total of $2 \cdot 5!=240$ words that end in $a$ or $b$.
\item To find the total, we must add the results from $(b)$ and $(c)$, but subtract out the number that get overcounted through PIE. Thus, we must subtract the number that start with $d$ or $e$ and end in $b$ or $a$. We have 2 options for the first and last letters, and 4 for the second, 3 for the third... for a total of $2\cdot2\cdot4! = 96$ words. Next, we subtract $96$ from the sum of $(a)$ and $(b)$, $240 + 240 = 480$. Therefore, we have $480 - 96 = 384$ words that both begin with $d$ or $e$ and end with $a$ or $b$.
\item This question asks for the complement of $(d)$: in $(d)$, we found how many words both begin with $d$ or $e$ and end with $b$ or $a$, whereas now we must find the words that do not fall into that category. Since we know how many words we have in total from $(a)$, we can subtract the words that do not fit the criteria from the total. Thus, we have $720 - 384 = 336$ words that neither begin with $d$ or $e$ nor end with $a$ or $b$.
\end{enumerate}
\end{proof}

\clearpage

\begin{theorem}
    Consider the recurrence relation $a_n = s a_{n-1}+d$ where $s\ne 1$.
    Prove that
    \[
        a_n = \left(a_0 + \frac{d}{s-1}\right) s^n - \frac{d}{s-1}
    \]
    is a solution. Then use this theorem to solve for a closed formula for the recurrence $a_n = 5 a_{n-1} + 3$ where $a_0 = 1$.
\end{theorem}

\begin{proof}
To show that the shown solution is actually a solution to the shown recurrence relation, we will substitute $a_{n-1}$ into the original recurrence relation. 

\begin{align*}
a_{n-1} &= (a_0 + \frac{d}{s-1})s^{n-1}-\frac{d}{s-1}
\end{align*}
This can then be substituted into $a_n = sa_{n-1} + d$.
\begin{align*}
a_n &= s((a_0 + \frac{d}{s-1})s^{n-1}-\frac{d}{s-1}) + d\\
a_n &= s((a_0 + \frac{d}{s-1})s^{n-1}) + \frac{-sd}{s-1} + d\\
a_n &= s \cdot s^{n-1}(a_0 + \frac{d}{s-1}) + \frac{-sd}{s-1} + d\\
a_n &= s^n(a_0 + \frac{d}{s-1}) + \frac{-sd}{s-1} + d\\
a_n &= s^n(a_0 + \frac{d}{s-1}) + \frac{-sd}{s-1} + \frac{d(s-1)}{s-1}\\
a_n &= s^n(a_0 + \frac{d}{s-1}) + \frac{-sd}{s-1} + \frac{sd - d}{s-1}\\
a_n &= s^n(a_0 + \frac{d}{s-1}) + \frac{-sd + sd - d}{s-1}\\
a_n &= s^n(a_0 + \frac{d}{s-1}) + \frac{-d}{s-1}\\
a_n &= (a_0 + \frac{d}{s-1})s^n - \frac{d}{s-1} 
\end{align*}
Thus, the solution can be used to find a closed-form solution to recurrence relations of the form $a_n = s a_{n-1}+d$ where $s\ne 1$.
\end{proof}
\clearpage
\begin{proof}
To solve the recurrence relation $a_n = 5 a_{n-1} + 3$ where $a_0 = 1$, we will find the closed-form equation using the formula $a_n = \left(a_0 + \frac{d}{s-1}\right) s^n - \frac{d}{s-1}$, since the relation follows the form $a_n = s a_{n-1}+d$. Using this formula and previously known values, we know that $a_0 = 1$, $d =3$, and $s=5$. Plugging these in and solving, we see that
\begin{align*}
a_n &= (1 + \frac{3}{5-1})\cdot5^n - \frac{3}{5-1}\\
a_n &= ( 1 + \frac{3}{4}) 5^n - \frac{3}{4}\\
a_n &= \frac{7}{4}5^n - \frac{3}{4}
\end{align*}
\end{proof}

\clearpage

\begin{conjecture}
    For all $n\ge 1$,
    \[
        \frac{1}{1\cdot 2} + \frac{1}{2\cdot 3} + \cdots + \frac{1}{n\cdot(n+1)} = \frac{n}{n+1}
    \]
\end{conjecture}

\begin{proof}
(By induction.)
Let $P(n)$ be the statement that for all $n \ge 1$, $\frac{1}{1\cdot 2} + \frac{1}{2\cdot 3} + \cdots + \frac{1}{n\cdot(n+1)} = \frac{n}{n+1}$. 

Base Case: For $n = 1$.
\begin{align*}
P(1) = \frac{1}{1 \cdot (1+1)} &= \frac{1}{1+1}\\
\frac{1}{2} &= \frac{1}{2}
\end{align*}
Since $P(1)$ is true, we can move on to the Inductive Hypothesis.

Inductive Hypothesis: Assume $P(k)$ is true for all $1 \le k < n$.

Inductive Step: Consider $k+1$.
\begin{align*}
P(k+1) = \frac{1}{1\cdot 2} + \frac{1}{2\cdot 3} + \cdots + \frac{1}{k\cdot(k+1)} + \frac{1}{(k+1) \cdot (k+1+1)}.
\end{align*}
This can be reduced to
\begin{align*}
\frac{k}{k+1} + \frac{1}{(k+1) \cdot (k+2)},
\end{align*}
which can be further reduced
\begin{align*}
&=\frac{k}{k+1}\cdot \frac{k+2}{k+2} + \frac{1}{(k+1) \cdot (k+2)}\\
&=\frac{k(k+2)}{(k+1)(k+2)} + \frac{1}{(k+1)(k+2)}\\
&=\frac{k^2+2k+1}{(k+1)(k+2)}\\
&=\frac{(k+1)^2}{(k+1)(k+2)}\\
&=\frac{k+1}{k+2}\\
& = \frac{k+1}{(k+1)+1}.
\end{align*}
Therefore, since $P(k)$ is true for all $1 \le k < n$, by strong induction we can say that $P(n)$ is true for all values of $n \ge 1$.
\end{proof}
\clearpage

For Theorem \ref{thm:modular-squares}, we will use the following definition.

\theoremstyle{definition}
\newtheorem{definition}{Definition}
\begin{definition}
    Let $a,b,\in\Z$ and $m\in \N$ with $m > 1$.
    We say that $a$ is \emph{congruent to $b$ modulo $m$} if $m|(a-b)$.
    We write $a \equiv b\mod m$.
\end{definition}

Thus, e.g., $11\equiv 3\mod 4$, since $4|11-3$, but $9\not\equiv 3\mod 4$ since $4\nmid 9-3$.

\begin{theorem}\label{thm:modular-squares}
    Suppose $a,b\in \Z$ and $m\in \N$ with $m > 1$ such that $a\equiv b\mod m$.
    Then $a^2 \equiv b^2\mod m$.
\end{theorem}

\begin{proof}
(By Contradiction) From Theorem 6, assume $a \equiv b$ mod $m$ and $a^2 \not \equiv b^2$ mod $m$.

Contradiction: Let $a=6$, $b=2$, and $m=4$. The statement $m|(a-b)$ is true since $4$ divides $6-2=4$. As for $m|(a^2-b^2)$, $4|(6^2-2^2)$ since $4$ divides $6^2-4^2=36-4=32$. Here is a clear contradiction. Therefore, by contradiction, we can conclude that if $a \equiv b$ mod $m$, then $a^2 \equiv b^2$ mod $m$ as well.
\end{proof}

\clearpage

\rubric

\end{document}